%%	build-queue:
%%	
%%	¦¦¦ very first run. No .ist files yet
%%	¦¦	index, citation/bibliography or glossary changed
%%	¦	every change apart from the above mentioned. Double run for labels and toc.
%%	
%%	¦¦¦xelatex
%%	¦¦makeglossaries
%%	¦¦makeindex
%%	¦¦bibtex
%%	¦xelatex
%%	¦xelatex
%%	

\RequirePackage[l2tabu,orthodox]{nag}
\documentclass[10pt,a4paper,titlepage,twoside,english, final]{zhawreprt}

\usepackage[T1]{fontenc}


\include{packages}
\if false
\include{glossaryentries}
\fi

\logofilename{images/logos/SoE/de/de-soe-cmyk.png}
\projecttype{PA}
\major{HS17 Studiengang Informatik}
\title{Validierung verschiedener Übersetzungsansätze}
\shorttitle{Spaghetti}
\author{}
\authors{Nicolas Hoferer, Daniel Einars}
\mainreferee{Marc Cieliebak}
\referee{Kurt Stockinger // Jan Milan Deriu}
\industrypartner{}
\extreferee{}
\setdate{\today}

\begin{document}

\maketitle

\chapter*{Abstract}\label{sec:Abstract}
\notes{\item Summary}
\text{This is just some normal text that goes here}

\chapter*{Preface}\label{sec:Preface}
\notes{\item Stellt den persönlichen Bezug zur Arbeit dar und spricht Dank aus.}
\text{thank-yous go here}
\makedeclarationoforiginality

\tableofcontents

\chapter{Introduction}\label{chp:Introduction}
\section{InitialPosition}\label{sec:InitialPosition}
\notes{
\item Nennt bestehende Arbeiten/Literatur zum Thema -> Literaturrecherche
\item Stand der Technik: Bisherige Lösungen des Problems und deren Grenzen
\item (Nennt kurz den Industriepartner und/oder weitere Kooperationspartner und dessen/deren Interesse am Thema Fragestellung)
}
\section{Task}\label{sec:Task}
\notes{
\item Formuliert das Ziel der Arbeit
\item Verweist auf die offizielle Aufgabenstellung des/der Dozierenden im Anhang
\item (Pflichtenheft, Spezifikation)
\item (Spezifiziert die Anforderungen an das Resultat der Arbeit)
\item (Übersicht über die Arbeit: stellt die folgenden Teile der Arbeit kurz vor)
\item (Angaben zum Zielpublikum: nennt das für die Arbeit vorausgesetzte Wissen)
\item (Terminologie: Definiert die in der Arbeit verwendeten Begriffe)
}

\chapter{Theoretical Principles}\label{chp:TheoreticalPrinciples}
Test this one here too, please
\setlistingCSharp
\begin{lstlisting}[linebackgroundcolor={\inlist{lstnumber}{3,4}}]
int getRandomNumber()
{
	return 4; // chosen by fair dice roll.
			  // guaranteed to be random.
}
\end{lstlisting}

\chapter{Method}\label{chp:Method}
\notes{
\item (Beschreibt die Grundüberlegungen der realisierten Lösung (Konstruktion/Entwurf) und die Realisierung als Simulation, als Prototyp oder als Software-Komponente)
\item (Definiert Messgrössen, beschreibt Mess- oder Versuchsaufbau, beschreibt und dokumentiert Durchführung der Messungen/Versuche)
\item (Experimente)
\item (Lösungsweg)
\item (Modell)
\item (Tests und Validierung)
\item (Theoretische Herleitung der Lösung)
}

\chapter{Results}\label{chp:Results}
\notes{\item (Zusammenfassung der Resultate)}

\chapter{Discussion and Prospects}\label{chp:DiscussionAndProspects}
Wie in \cite[Kapitel~2, Seite~215]{DahmenReusken2008} nachzulesen, gibt es sogenannte Gleichungen\index{Gleichung}.\gls{hrz}\gls{elitism}\gls{ohm}
\notes{
\item Bespricht die erzielten Ergebnisse bezüglich ihrer Erwartbarkeit, Aussagekraft und Relevanz
\item Interpretation und Validierung der Resultate
\item Rückblick auf Aufgabenstellung, erreicht bzw. nicht erreicht
\item Legt dar, wie an die Resultate (konkret vom Industriepartner oder weiteren Forschungsarbeiten; allgemein) angeschlossen werden kann; legt dar, welche Chancen die Resultate bieten
}

\chapter{Index}\label{chp:Index}
\bibliography{reference}\label{sec:Bibliography}
\newpage
\printglossary[title=Glossary]\label{sec:Glossar}
\newpage
\listoffigures\label{sec:ListOfFigures}
\newpage
\listoftables\label{sec:ListOfTables}
\newpage
\lstlistoflistings\label{sec:ListOfListings}
\newpage
\printglossary[title=Symbol Glossary, type=symbols]\label{sec:SymbolGlossary}
\newpage
\printglossary[title=Acronym Glossary,type=\acronymtype]\label{sec:AcronymGlossary}
\newpage
\printindex\label{sec:Index}

\appendix
\chapter{Appendix}\label{chp:Appendix}
\section{Projektmanagement}\label{sec:Projectmanagement}
\notes{
\item Offizielle Aufgabenstellung, Projektauftrag
\item (Zeitplan) 
\item (Besprechungsprotokolle oder Journals)
}
\section{Final Words}\label{sec:Others}
\notes{
\item CD mit dem vollständigen Bericht als pdf-File inklusive Film- und Fotomaterial
\item (Schaltpläne und Ablaufschemata)
\item (Spezifikationen u. Datenblätter der verwendeten Messgeräte und/oder Komponenten)
\item (Berechnungen, Messwerte, Simulationsresultate)
\item (Stoffdaten)
\item (Fehlerrechnungen mit Messunsicherheiten)
\item (Grafische Darstellungen, Fotos)
\item (Datenträger mit weiteren Daten(z. B. Software-Komponenten) inkl. Verzeichnis der auf diesem Datenträger abgelegten Dateien)
\item (Softwarecode)
}
\end{document}
