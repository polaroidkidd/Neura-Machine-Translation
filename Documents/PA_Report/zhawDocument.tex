%%	build-queue:
%%	
%%	¦¦¦ very first run. No .ist files yet
%%	¦¦	index, citation/bibliography or glossary changed
%%	¦	every change apart from the above mentioned. Double run for labels and toc.
%%	
%%	¦¦¦xelatex
%%	¦¦makeglossaries
%%	¦¦makeindex
%%	¦¦bibtex
%%	¦xelatex
%%	¦xelatex
%%	

\RequirePackage[l2tabu,orthodox]{nag}
\documentclass[10pt,a4paper,titlepage,twoside,english, final]{zhawreprt}

\usepackage[T1]{fontenc}


	\setdefaultlanguage{german}
\RequirePackage{lscape}
\usepackage{pgfgantt}
\usepackage{xcolor}
	\definecolor{lgray}{RGB}{250,250,250}
	\definecolor{lgreen}{RGB}{63,127,95}
	\definecolor{lred}{RGB}{127,0,85}
	\definecolor{lblue}{RGB}{42,0,255}
\usepackage{listings}
	\lstdefinestyle{basestyle}{
		basicstyle=\small\ttfamily,
		breakatwhitespace = true,
		tabsize = 4,
		frame = double,
		numbers = left,
		numbersep = 10pt,
		numberstyle = {\tiny\emptyaccsupp},
%		firstnumber = auto,
		numberblanklines = true,
		captionpos = b,
		columns = fullflexible,
		extendedchars = true,
		float = ht,
		showtabs = false,
		showspaces=false,
		showstringspaces=false,
		breaklines=true,
%		prebreak=\Righttorque
		backgroundcolor=\color{lgray},
		keywordstyle=\color{lred}\bfseries,
		commentstyle=\color{lgreen}\ttfamily,
%		morekeywords={printstr, printhexln},
		stringstyle=\color{lblue},
		xleftmargin = \fboxsep,
		xrightmargin = -6pt,
		showstringspaces=true,
	}
	\newcommand{\setlistingCSharp}{
		\lstset{
		style = basestyle,
		language = [Sharp]C,
		%otherkeywords = {*,<,>,=,;,\{,\}},
		%deletekeywords = {...},
	}}
	\newcommand{\setlistingCpp}{
		\lstset{
		style = basestyle,
		language = C++,
		%otherkeywords = {*,<,>,=,;,\{,\}},
		%deletekeywords = {...},
	}}
	\newcommand{\setlistingJava}{
		\lstset{
		style = basestyle,
		language = Java,
		%otherkeywords = {*,<,>,=,;,\{,\}},
		%deletekeywords = {...},
	}}
	\newcommand{\setlistingLaTeX}{
		\lstset{
		style = basestyle,
		language = TeX,
		%otherkeywords = {*,<,>,=,;,\{,\}},
		%deletekeywords = {...},
	}}
	\newcommand{\setlistingMatlab}{
		\lstset{
		style = basestyle,
		language = Matlab,
		otherkeywords = {methods,enumeration,properties,classdef,Sealed,Abstract},
		%deletekeywords = {...},
	}}
\usepackage{xstring}
	\newcommand{\inlist}[2]{
		\IfSubStr{,#2,}{,\arabic{#1},}{\color{lgray!95!blue}}{\color{lgray}}
	}
\usepackage{lstlinebgrd}
	\makeatletter
	\renewcommand{\lst@linebgrd}{%
	\ifx\lst@linebgrdcolor\empty\else
		\rlap{%
			\lst@basicstyle
			\color{lgray}
			\lst@linebgrdcolor{%
				\kern-\dimexpr\lst@linebgrdsep\relax%
				\lst@linebgrdcmd{\lst@linebgrdwidth}{\lst@linebgrdheight}{\lst@linebgrddepth }%
			}%
		}%
	\fi}
	\makeatother
\usepackage[space=true]{accsupp}
	\newcommand\emptyaccsupp[1]{\BeginAccSupp{ActualText={}}#1\EndAccSupp{}}
\usepackage{rotating}
\usepackage{mathptmx}
\usepackage{amssymb}
\usepackage{textcomp}
\usepackage[squaren]{SIunits}
\usepackage{amsmath}
\usepackage{amsfonts}
\usepackage{amssymb}
\usepackage[toc,page]{appendix}
\usepackage{fontspec}
	\setmainfont{Arial} % sets the roman font
	\setsansfont{Arial} % sets the sans-sérif font
	\setmonofont{Arial} % sets the monospace font
\usepackage{microtype}
\usepackage{colortbl}
\usepackage{tabularx}
\usepackage{longtable}
\usepackage{pgf,tikz}
	\usetikzlibrary{shapes}
	\usetikzlibrary{shapes.geometric}
	\usetikzlibrary{shapes.arrows}
	\usetikzlibrary{positioning}
	\usetikzlibrary{fit}
	\usetikzlibrary{calc}
	\usetikzlibrary{patterns}
\usepackage{pgfplots}
	\pgfplotsset{compat=1.13}
\usepackage{array}
\usepackage{natbib}
	\bibliographystyle{agsm}
\usepackage{usecases}
\usepackage{footnote}
	\makesavenoteenv{description}
\usepackage{makeidx}
	\makeindex
\usepackage[nottoc]{tocbibind}
\makeatletter
	\if@inltxdoc\else
	  \renewenvironment{theindex}%
	    {\if@twocolumn
	       \@restonecolfalse
	     \else
	       \@restonecoltrue
	     \fi
	     \if@bibchapter
	        \if@donumindex
	          \refstepcounter{section}
	          \twocolumn[\vspace*{2\topskip}%
	                     \@makechapterhead{\indexname}]%
	          \addcontentsline{toc}{section}{\protect\numberline{\thesection}\indexname}
	          \sectionmark{\indexname}
	        \else
	          \if@dotocind
	            \twocolumn[\vspace*{2\topskip}%
	                       \@makeschapterhead{\indexname}]%
	            \prw@mkboth{\indexname}
	            \addcontentsline{toc}{section}{\protect\numberline{\thesection}\indexname}
	          \else
	            \twocolumn[\vspace*{2\topskip}%
	                       \@makeschapterhead{\indexname}]%
	            \prw@mkboth{\indexname}
	          \fi
	        \fi
	     \else
	        \if@donumindex
	          \twocolumn[\vspace*{-1.5\topskip}%
	                     \@nameuse{\@tocextra}{\indexname}]%
	          \csname \@tocextra mark\endcsname{\indexname}
	        \else
	          \if@dotocind
	            \twocolumn[\vspace*{-1.5\topskip}%
	                       \toc@headstar{\@tocextra}{\indexname}]%
	            \prw@mkboth{\indexname}
	            \addcontentsline{toc}{\@tocextra}{\indexname}
	          \else
	            \twocolumn[\vspace*{-1.5\topskip}%
	                       \toc@headstar{\@tocextra}{\indexname}]%
	            \prw@mkboth{\indexname}
	          \fi
	        \fi
	     \fi
	   \thispagestyle{plain}\parindent\z@
	   \parskip\z@ \@plus .3\p@\relax
	   \let\item\@idxitem}
	   {\if@restonecol\onecolumn\else\clearpage\fi}
	\fi
\makeatother
\usepackage{etoolbox}
\makeatletter
	\renewcommand\listoftables{%
	    \section{Tabellenverzeichnis}%
	    \@mkboth{\MakeUppercase\listtablename}%
	        {\MakeUppercase\listtablename}%
	    \@starttoc{lot}%
	}
	\renewcommand\listoffigures{%
	    \section{Abbildungsverzeichnis}%
	    \@mkboth{\MakeUppercase\listfigurename}%
	        {\MakeUppercase\listfigurename}%
	    \@starttoc{lof}%
	}
	\renewcommand\lstlistoflistings{%
	    \section{Listingverzeichnis}%
	    \@mkboth{\MakeUppercase\listfigurename}%
	        {\MakeUppercase\listfigurename}%
	    \@starttoc{lol}%
	}
	\renewenvironment{thebibliography}[1]
	     {\section{\bibname}% <-- this line was changed from \chapter* to \section*
	      \@mkboth{\MakeUppercase\bibname}{\MakeUppercase\bibname}%
	      \list{\@biblabel{\@arabic\c@enumiv}}%
	           {\settowidth\labelwidth{\@biblabel{#1}}%
	            \leftmargin\labelwidth
	            \advance\leftmargin\labelsep
	            \@openbib@code
	            \usecounter{enumiv}%
	            \let\p@enumiv\@empty
	            \renewcommand\theenumiv{\@arabic\c@enumiv}}%
	      \sloppy
	      \clubpenalty4000
	      \@clubpenalty \clubpenalty
	      \widowpenalty4000%
	      \sfcode`\.\@m}
	     {\def\@noitemerr
	       {\@latex@warning{Empty `thebibliography' environment}}%
	      \endlist}
\makeatother
	
	\renewcommand\appendixname{Appendix}
\makeatletter
	\renewenvironment{theindex}{
		\renameindex
		\let\ps@plainorig\ps@plain
		\let\ps@plain\ps@scrheadings
		\if@twocolumn
			\@restonecolfalse
		\else
			\@restonecoltrue
		\fi
		\columnseprule \z@
		\columnsep 35\p@
		\twocolumn[\section{\indexname}]%
		\@mkboth{\MakeUppercase\indexname}{\MakeUppercase\indexname}%
		\thispagestyle{plain}\parindent\z@
		\parskip\z@ \@plus .3\p@\relax
	\let\item\@idxitem}
	{\if@restonecol\onecolumn\else\clearpage\fi}
\makeatother

\usepackage{varioref}
\usepackage[colorlinks=true, linkcolor=black, citecolor=black, plainpages=false, unicode, pdfencoding=auto ,backref=page]{hyperref}
\usepackage{cleveref}
\usepackage[automake,acronym,numberedsection]{glossaries-extra}
	\renewcommand*{\glspostdescription}{}
	\newglossary[slg]{symbols}{sym}{sbl}{Symbolverzeichnis}
	\makeglossaries
	\loadglsentries{glossaryentries.tex}
	\pdfstringdefDisableCommands{\let\textenglish\@firstofone\let\textgerman\@firstofone}
	\makeatletter
	\renewcommand*{\@@glossarysec}{section}
	\makeatother
\if false
%\newcommand{\FullGlossaryEntry}[7]{%Abkz., Typ, Name, Name(plural), Beschr., Beschr.(plural), Erkl.
%\newglossaryentry{#1}{type=\acronymtype,
%name={\textit{#3}},
%plural={\textit{#4}},
%description={\textit{#5}},
%first={\textit{#5}(nachfolgend #3)\glsadd{_#1}},
%firstplural={\textit{#6}(nachfolgend \glsentryplural{#4})},
%see=[Glossar:]{_#1}
%}
%\longnewglossaryentry{_#1}{type=#2,
%name={\textit{#5}},
%plural={\textit{#6}}
%}{\hspace{0pt}\\#7}
%}


% Abkuerzungen
\newacronym{jlu}{JLU}{Justus-Liebig-Universität}
\newacronym{hrz}{HRZ}{Hochschulrechenzentrum}
\newacronym[plural=LEDs, longplural={light-emitting diodes}]{led}{LED}{light-emitting diode}
\newacronym[plural=EEPROMs, longplural={electrically erasable programmable read-only memories}]{eeprom}{EEPROM}{electrically erasable programmable read-only memory}

%  Glossareintraege
\newglossaryentry{culdesac}{name=cul-de-sac, description={passage or street closed at one end}, plural=culs-de-sac}
\newglossaryentry{elite}{name={é}lite, description={select group or class}, sort=elite}
\newglossaryentry{elitism}{name={é}litism, description={advocacy of dominance by an \gls{elite}}, sort=elitism}
\newglossaryentry{attache}{name=attaché, description={person with special diplomatic responsibilities}}

% Eintraege für Symbolliste
\newglossaryentry{ohm}{type=symbols, name={\ensuremath{\Omega}}, sort=Ohm, symbol={\ensuremath{\Omega}}, description={unit of electrical resistance}}
\newglossaryentry{angstrom}{type=symbols, name={\AA}, sort=angström, symbol={\AA}, description={non-SI unit of length}}
%
%\FullGlossaryEntry{zhaw}{main}%
%{ZHaW}{}%
%{Zürcher Hochschule der angewandten Wissenschaften}{}%
%{Name der Hochschule meines Vertrauens.}
%
%\FullGlossaryEntry{mnmt1}{main}%
%{MNMT1}{}%
%{Mathematik: Numerik für Maschinentechnik 1}{}%
%{Erstes Numerik-Modul im Studiengang Maschinentechnik an der \gls{zhaw}. Beinhaltet Taylor- und Fourier-Reihen, Numerik gewöhnlicher Differentialgleichungen anhand des Euler-, Taylor- und Runge-Kutta-Verfahren, Numerik nichtlinearer Differentialgleichungen, Lagrange- und Newton-Interpolation, Splines und Ausgleichsrechnung.}
%
%\FullGlossaryEntry{mnmt2}{main}%
%{MNMT2}{}%
%{Mathematik: Numerik für Maschinentechnik 2}{}%
%{Zweites Numerik-Modul im Studiengang Maschinentechnik an der \gls{zhaw}. Beinhaltet Randwertprobleme, numerische Differentiation und Integration, Numerik partieller Differentialgleichungen anhand der \gls{fdm} und \gls{fem}, Stabilität von numerischen Verfahren zur Lösung von gewöhnlichen Differentialgleichungen, Schrittweitensteuerung und implizite Runge-Kutta-Verfahren.}
%
%\FullGlossaryEntry{fdm}{main}%
%{FDM}{}%
%{Finite Differenzen Methode}{}%
%{Numerisches Verfahren zur Lösung von \glspl{rwp}}
%
%\FullGlossaryEntry{fem}{main}%
%{FEM}{}%
%{Finite Elemente Methode}{}%
%{Numerisches Verfahren zur Lösung von \glspl{rwp}}
%
%\FullGlossaryEntry{rwp}{main}%
%{RWP}{RWPs}%
%{Randwertproblem}{Randwertprobleme}%
%{Ein Differentialproblem, bei dem diskrete Funktionswerte einer bestimmten Ordnung an beliebigen Stellen gegeben sind.}
%
%\FullGlossaryEntry{awp}{main}%
%{AWP}{AWPs}%
%{Anfangswertproblem}{Anfangswertprobleme}%
%{Ein Differentialproblem, bei dem alle diskrete Funktionswerte an einer Anfangsstelle gegeben sind.}
%
%\FullGlossaryEntry{dgl}{main}%
%{DGL}{DGLn}%
%{Differentialgleichung}{Differentialgleichungen}%
%{Eine Gleichung, bei der Ableitungen beliebiger Ordnung einer gesuchten Funktion vorkommen.}
%
%\FullGlossaryEntry{ode}{main}%
%{ode}{ode's}%
%{ordinary differential equation}{ordinary differential equations}%
%{Englische Bezeichnung für \gls{dgl}}
%
%
%\FullGlossaryEntry{dgls}{main}%
%{DGLS}{DGLSe}%
%{Differentialgleichungssystem}{Differentialgleichungssysteme}%
%{Ein System von Differentialgleichungen}
%
%\FullGlossaryEntry{ods}{main}%
%{ods}{ods's}%
%{ordinary differential system}{ordinary differential systems}%
%{Englische Bezeichnung für \gls{dgls}}
\fi

\logofilename{images/logos/SoE/de/de-soe-cmyk.png}
\projecttype{PA}
\major{HS17 Degree: Computer Science}
\title{Techniques for ML-Assisted Language Translation}
\shorttitle{Spaghetti}
\author{}
\authors{Nicolas Hoferer

Daniel Einars}
\mainreferee{Marc Cieliebak}
\referee{Kurt Stockinger

Jan Milan Deriu}
\industrypartner{}
\extreferee{}
\setdate{30.10.2017}

\begin{document}

\maketitle

\chapter*{Abstract}\label{sec:Abstract}
\notes{\item Summary}
\text{This is just some normal text that goes here}

\chapter*{Preface}\label{sec:Preface}
\notes{\item Stellt den persönlichen Bezug zur Arbeit dar und spricht Dank aus.}
\text{thank-yous go here}
\makedeclarationoforiginality

\tableofcontents

\chapter{Introduction}\label{chp:Introduction}
\begin{itemize}
\item machines becoming better at processing human language (accuracy)
\item conversation with machines are possible to a limited degree
\item information retrieval through voice recognition still a challenge due to
\begin{itemize}
\item attention to correct words
\item database structures
\item multiple ways to ask for identical information
\end{itemize}
\item multiple solutions proposed
\begin{itemize}
\item KBQA: Learning Question Answering over QA Corpora and Knowledge Bases
\item Eric, Manning - 2017 - Key-Value Retrieval Networks for Task-Oriented Dialogue - With Highlights
\end{itemize}
\item Asking your Assistant (Google, Siri or S-Voice) weather you have an appointment tomorrow and ask follow-up questions about this appointment is currently not possible (due to above challenges but could be if these papers prove implementable
\end{itemize}



\section{Initial Position}\label{sec:InitialPosition}
\begin{itemize}
\item No Response from KBQA for Code
\item Refusal to share code from Manning
\item Ultimate new goal: Implement Manning's solution without his code
\end{itemize}
\notes{
\item Nennt bestehende Arbeiten/Literatur zum Thema -> Literaturrecherche
\item Stand der Technik: Bisherige Lösungen des Problems und deren Grenzen
\item (Nennt kurz den Industriepartner und/oder weitere Kooperationspartner und dessen/deren Interesse am Thema Fragestellung)
}
\section{Task}\label{sec:Task}
\begin{itemize}
\item Small Steps
\begin{itemize}
\item implement seq2seq network for translation
\begin{itemize}
\item implement char-based
\item implement word-based
\item try multiple different implementations (reversed-input, multiple LSTMs) and compare against each other
\item get decent results on both and move on
\end{itemize}
\item implement seq2seq with attention
\begin{itemize}
\item attempt various attention mechanism
\end{itemize}
\end{itemize}
\item One Large Step
\begin{itemize}
\item map best working models and tools to KBQA and get better results than Stanford
\item Rub better results in Eric's face.
\item Profit.
\end{itemize}
\end{itemize}
\notes{
\item Formuliert das Ziel der Arbeit
\item Verweist auf die offizielle Aufgabenstellung des/der Dozierenden im Anhang
\item (Pflichtenheft, Spezifikation)
\item (Spezifiziert die Anforderungen an das Resultat der Arbeit)
\item (Übersicht über die Arbeit: stellt die folgenden Teile der Arbeit kurz vor)
\item (Angaben zum Zielpublikum: nennt das für die Arbeit vorausgesetzte Wissen)
\item (Terminologie: Definiert die in der Arbeit verwendeten Begriffe)
}

\chapter{Theoretical Principles}\label{chp:TheoreticalPrinciples}
Test this one here too, please
\setlistingCSharp
\begin{lstlisting}[linebackgroundcolor={\inlist{lstnumber}{3,4}}]
int getRandomNumber()
{
	return 4; // chosen by fair dice roll.
			  // guaranteed to be random.
}
\end{lstlisting}
\section{Definitions}\label{sec:Definitions}
\begin{itemize}
\item Hypothesis
\item Reference
\end{itemize}
\section{Recurrent Neural Networks}\label{sec:Recurrent Neural Networks}
\begin{itemize}
\item Standard Neural Networks
\item Recurrent Neural Networks
\begin{itemize}
\item Problems
\item Solutions
\end{itemize}
\end{itemize}
\section{Seq2Seq}\label{sec:Seq2Seq}
\begin{itemize}
\item encoder
\item decoder
\end{itemize}
\section{Attention}\label{sec:Attention Mechanisms}
\begin{itemize}
\item Mechanisms
\end{itemize}
\section{Performance Evaluation}\label{sec:Performance Evaluation}
\begin{itemize}
\item Translation
\begin{itemize}
\item Bleu
\item others
\end{itemize}
\item KB-Retrieval
\begin{itemize}
\item Bleu
\item sent2vec
\end{itemize}
\end{itemize}

\chapter{Method}\label{chp:Method}
\notes{
\item (Beschreibt die Grundüberlegungen der realisierten Lösung (Konstruktion/Entwurf) und die Realisierung als Simulation, als Prototyp oder als Software-Komponente)
\item (Definiert Messgrössen, beschreibt Mess- oder Versuchsaufbau, beschreibt und dokumentiert Durchführung der Messungen/Versuche)
\item (Experimente)
\item (Lösungsweg)
\item (Modell)
\item (Tests und Validierung)
\item (Theoretische Herleitung der Lösung)
}

\chapter{Results}\label{chp:Results}
\notes{\item (Zusammenfassung der Resultate)}

\chapter{Discussion and Prospects}\label{chp:DiscussionAndProspects}
Wie in \cite[Kapitel~2, Seite~215]{DahmenReusken2008} nachzulesen, gibt es sogenannte Gleichungen\index{Gleichung}.\gls{hrz}\gls{elitism}\gls{ohm}
\notes{
\item Bespricht die erzielten Ergebnisse bezüglich ihrer Erwartbarkeit, Aussagekraft und Relevanz
\item Interpretation und Validierung der Resultate
\item Rückblick auf Aufgabenstellung, erreicht bzw. nicht erreicht
\item Legt dar, wie an die Resultate (konkret vom Industriepartner oder weiteren Forschungsarbeiten; allgemein) angeschlossen werden kann; legt dar, welche Chancen die Resultate bieten
}

\chapter{Index}\label{chp:Index}
\bibliography{reference}\label{sec:Bibliography}
\newpage
\printglossary[title=Glossary]\label{sec:Glossar}
\newpage
\listoffigures\label{sec:ListOfFigures}
\newpage
\listoftables\label{sec:ListOfTables}
\newpage
\lstlistoflistings\label{sec:ListOfListings}
\newpage
\printglossary[title=Symbol Glossary, type=symbols]\label{sec:SymbolGlossary}
\newpage
\printglossary[title=Acronym Glossary,type=\acronymtype]\label{sec:AcronymGlossary}
\newpage
\printindex[title=Index]\label{sec:Index}

\appendix
\chapter{Appendix}\label{chp:Appendix}
\section{Projektmanagement}\label{sec:Projectmanagement}
\notes{
\item Offizielle Aufgabenstellung, Projektauftrag
\item (Zeitplan) 
\item (Besprechungsprotokolle oder Journals)
}
\section{Final Words}\label{sec:Others}
\notes{
\item CD mit dem vollständigen Bericht als pdf-File inklusive Film- und Fotomaterial
\item (Schaltpläne und Ablaufschemata)
\item (Spezifikationen u. Datenblätter der verwendeten Messgeräte und/oder Komponenten)
\item (Berechnungen, Messwerte, Simulationsresultate)
\item (Stoffdaten)
\item (Fehlerrechnungen mit Messunsicherheiten)
\item (Grafische Darstellungen, Fotos)
\item (Datenträger mit weiteren Daten(z. B. Software-Komponenten) inkl. Verzeichnis der auf diesem Datenträger abgelegten Dateien)
\item (Softwarecode)
}
\end{document}
